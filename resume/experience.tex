\cvsection{SELECT RESEARCH EXPERIENCE}
\begin{cventries}
  \cventry
    {Advisors: Professor Chris Callison-Burch, Professor Jean Gallier}
    {Using Paraphrases to Cluster and Order Adjectives by Intensity}
    {Philadelphia, PA}
    {Summer 2016 - Present}
    {
      \begin{cvitems}
        % 
        \item {Major part of a multiple year study to gather adjectives in the paraphrase database, cluster them based on semantic similarity, and rank adjectives within each cluster by emotional intensity, with application to intelligent dialogue agents  (http://scalar-adjectives.herokuapp.com/).}
        % 
        \item {Led the ranking effort by first successfully reproducing the state of the art study by Bansal and de Melo (2013), then improved it by 7.5\% (measured by Kendall's tau score) on de Melo’s test set, and suppressed de Melo by over 400\% on new test set procured from Amazon Mechanical Turks.}
        % 
        \item {Researched, developed and tested multiple ranking methods including: page rank, personalized page rank, various integer linear programming formulations, elastic net regression, logistic regression with l1 and l2 penalties, and incorporated prior information using beta-binomial model.}
        % 
      \end{cvitems}
    }
  \cventry
    {Advisor: Professor Lyle Ungar}
    {Deep Reinforcement Learning with Recurrent Encoder-Decoder for Conversation}
    {Philadelphia, PA}
    {Spring 2017}
    {
      \begin{cvitems}
        \item {Collaborated closely with a Ph.D. student to develop an intelligent agent capable of open domain conversation.}
        \item {Implemented and trained hierarchical recurrent neural net to maintain conversation history, and used deep reinforcement learning to incorporate the future outcome of the conversation in current word choice.}
        \item {Weekly reading group discussing select chapters from the “deeplearningbook” and critiqued the latest papers in the domain.}
      \end{cvitems}
    }
  \cventry
    {Advisor: Professor Chris Callison-Burch}
    {Bilingual Lexicon Induction with Deep Convolutional Neural Networks}
    {Philadelphia, PA}
    {Spring 2017}
    {
      \begin{cvitems}
        % 
        \item {Part of a three year effort to compile, pre-process, and evaluate one of the world’s largest collection of bilingual dictionaries pivoting through similar images (100 languages, 100 images per word, 25 terabytes of data) outside of those held by private entities such as Google.}
        % 
        \item {Constructed lexical representation of words in multiple languages using feature extraction layer of a deep convolutional neural network (AlexNet) for the purpose of automatic machine translation.}
        % 
        \item {Co-authored paper "A Large Multilingual Corpus for Learning Translation from Images" that was submitted to EMNLP 2017.}
      \end{cvitems} 
    }
  \cventry
    {Advisor: Professor Chris Callison-Burch}
    {Learning Translations via Matrix Completion}
    {Philadelphia, PA}
    {Spring 2017}
    {
      \begin{cvitems}
         \item {Part of a collaborative group comprised of Ph.D.’s, post-doctoral student, and visiting scholar to create a common framework for machine translation leveraging multiple sources of information, each of which incomplete and noisy.}
        \item {Participated in weekly meetings and co-authored final paper submitted and accepted to EMNLP 2017 (26\% acceptance rate).}
      \end{cvitems}
    }
  \cventry
    {Advisor: Professor Kostas Daniilidis}
    {Dense 3D Scene Reconstruction and Visual Odometry in Laparoscopic Video}
     % Augmented Reality in Surgery}
    {Philadelphia, PA}
    {Spring 2014 - Fall 2014}
    {
      \begin{cvitems}
        % 
        \item {Researched, developed, and trained convolutional neural net for human tissue segmentation in low resolution videos with the goal of assisting surgeons in augmented reality.}
        % 
        \item {Constructed training and test set and manually labeled tissue classes.}
        % 
      \end{cvitems}
    }
\end{cventries}
